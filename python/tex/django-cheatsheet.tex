\documentclass[10pt,english,landscape, a4]{article}
\usepackage{cpp}


\pdfinfo{
  /Title (C++ STL Container Cheatsheet.pdf)
  /Creator (Abdelkader Amar)
  /Author (Abdelkader Amar)
  /Subject (C++)
  /Keywords (C++, STL, Programming, Latex)}



\begin{document}
\raggedright\

\begin{center}
  \cheatsheettitle{Django Cheatsheet
}
\end{center}
%\vspace{0.5cm}

\footnotesize
\begin{multicols}{3}
  \raggedcolumns
  \noindent    %<---- here


{\centering\section{The Basics}}

\cppitem{Start the server}
\begin{minted}
  [frame=single, rulecolor=blue, framesep=1mm, baselinestretch=1, fontsize=\footnotesize]{cpp}
python manage.py runserver
python manage.py runserver 8080

\end{minted}

\cppitem{Create an app}
\begin{minted}
  [frame=single, rulecolor=blue, framesep=1mm, baselinestretch=1, fontsize=\footnotesize]{cpp}
python manage.py startapp app_name

\end{minted}

\cppitem{Create database table}
\begin{minted}
  [frame=single, rulecolor=blue, framesep=1mm, baselinestretch=1, fontsize=\footnotesize]{cpp}
python manage.py migrate

\end{minted}

\cppitem{Apply model change}
\begin{minted}
  [frame=single, rulecolor=blue, framesep=1mm, baselinestretch=1, fontsize=\footnotesize]{cpp}
python manage.py makemigrations app_name

\end{minted}

\cppitem{Show generate SQL}
\begin{minted}
  [frame=single, rulecolor=blue, framesep=1mm, baselinestretch=1, fontsize=\footnotesize]{cpp}
python manage.py sqlmigrate app_name migration_name

\end{minted}

{\centering\section{Admin}}

\cppitem{Create a super user}
\begin{minted}
  [frame=single, rulecolor=blue, framesep=1mm, baselinestretch=1, fontsize=\footnotesize]{cpp}
python manage.py createsuperuser

\end{minted}

\cppitem{Add a model to admin interface (app/admin.py)}
\begin{minted}
  [frame=single, rulecolor=blue, framesep=1mm, baselinestretch=1, fontsize=\footnotesize]{cpp}
admin.site.register(model_name)

\end{minted}

\cppitem{Filter with a list}
\begin{minted}
  [frame=single, rulecolor=blue, framesep=1mm, baselinestretch=1, fontsize=\footnotesize]{cpp}
# Filter the table Exchange based on a list of MIC codes
exchanges = Exchange.objects.filter(mic_code__in=exchanges_mics)

\end{minted}

\cppitem{Get media url prefix}
\begin{minted}
  [frame=single, rulecolor=blue, framesep=1mm, baselinestretch=1, fontsize=\footnotesize]{cpp}
<a href="{{ pdf_file }}">Download</a>

\end{minted}

\cppitem{Configuring media directory}
\begin{minted}
  [frame=single, rulecolor=blue, framesep=1mm, baselinestretch=1, fontsize=\footnotesize]{cpp}
# in settings.py
MEDIA_ROOT = 'media/generated'
MEDIA_URL = 'media/'
# in site/urls.py
from django.conf.urls.static import static
urlpatterns = [
    # ...
] + static(settings.MEDIA_URL, document_root=settings.MEDIA_ROOT)

\end{minted}

\cppitem{Query operators}
\begin{minted}
  [frame=single, rulecolor=blue, framesep=1mm, baselinestretch=1, fontsize=\footnotesize]{cpp}
__exact         __iexact
__contains      __icontains
__in
__gt            __gte           __lt        __lte
__startswith    __istartswith   __endswith  __iendswith
__range
__date          __year          __iso_year
__month         __day
__week          __week_day      __iso_week_day
__quarter
__time
__hour          __minute        __second
__isnull
__regex
__iregex

\end{minted}


\end{multicols}

\end{document}
