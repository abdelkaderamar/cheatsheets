\documentclass[10pt,english,landscape, a4]{article}
\usepackage{cpp}


\pdfinfo{
  /Title (C++ STL Container Cheatsheet.pdf)
  /Creator (Abdelkader Amar)
  /Author (Abdelkader Amar)
  /Subject (C++)
  /Keywords (C++, STL, Programming, Latex)}



\begin{document}
\raggedright\

\begin{center}
  \cheatsheettitle{Django Cheatsheet
}
\end{center}
%\vspace{0.5cm}

\footnotesize
\begin{multicols}{3}
  \raggedcolumns
  \noindent    %<---- here


\cppitem{Create a new project}
\begin{minted}
  [frame=single, rulecolor=blue, framesep=1mm, baselinestretch=1, fontsize=\footnotesize]{cpp}
django-admin startproject myproject
# create a project in the current directory
django-admin startproject myproject
# create a project in a different directory
django-admin startproject myproject project_dir

\end{minted}

\cppitem{Filter with a list}
\begin{minted}
  [frame=single, rulecolor=blue, framesep=1mm, baselinestretch=1, fontsize=\footnotesize]{cpp}
# Filter the table Exchange based on a list of MIC codes
exchanges = Exchange.objects.filter(mic_code__in=exchanges_mics)

\end{minted}

\cppitem{Get media url prefix}
\begin{minted}
  [frame=single, rulecolor=blue, framesep=1mm, baselinestretch=1, fontsize=\footnotesize]{cpp}
<a href="{{ pdf_file }}">Download</a>

\end{minted}

\cppitem{Configuring media directory}
\begin{minted}
  [frame=single, rulecolor=blue, framesep=1mm, baselinestretch=1, fontsize=\footnotesize]{cpp}
# in settings.py
MEDIA_ROOT = 'media/generated'
MEDIA_URL = 'media/'
# in site/urls.py
from django.conf.urls.static import static
urlpatterns = [
    # ...
] + static(settings.MEDIA_URL, document_root=settings.MEDIA_ROOT)

\end{minted}


\end{multicols}

\end{document}
