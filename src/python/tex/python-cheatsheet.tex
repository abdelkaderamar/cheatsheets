\documentclass[10pt,english,landscape, a4]{article}
\usepackage{cpp}


\pdfinfo{
  /Title (C++ STL Container Cheatsheet.pdf)
  /Creator (Abdelkader Amar)
  /Author (Abdelkader Amar)
  /Subject (C++)
  /Keywords (C++, STL, Programming, Latex)}



\begin{document}
\raggedright\

\begin{center}
  \cheatsheettitle{Python cheatsheet
}
\end{center}
%\vspace{0.5cm}

\footnotesize
\begin{multicols}{3}
  \raggedcolumns
  \noindent    %<---- here


\cppitem{Sort a list of objects}
\begin{minted}
  [frame=single, rulecolor=blue, framesep=1mm, baselinestretch=1, fontsize=\footnotesize]{cpp}
class stock:
    def __init__(self, name, ticker, isin):
self.name = name
self.ticker = ticker
self.isin = isin
sorted_stocks=sorted(stocks, key=lambda s: s.ticker, reverse=True)
stocks.sort(key=lambda s : s.ticker, reverse=True)

\end{minted}

\cppitem{Sort a dictionary by key}
\begin{minted}
  [frame=single, rulecolor=blue, framesep=1mm, baselinestretch=1, fontsize=\footnotesize]{cpp}
sorted_stocks_dict = {key: val for key, val     
    in sorted(stocks_dict.items(), key=lambda kvp : kvp[0])}

\end{minted}

\cppitem{Sort a dictionary by value (define \_\_lt\_\_ or \_\_gt\_\_)}
\begin{minted}
  [frame=single, rulecolor=blue, framesep=1mm, baselinestretch=1, fontsize=\footnotesize]{cpp}
class stock:
    def __init__(self, name, ticker, isin):
self.name = name
self.ticker = ticker
self.isin = isin
    def __lt__(self, other):
return self.name < other.name
sorted_stocks_dict = {key: val for key, val     
    in sorted(stocks_dict.items(), key=lambda kvp : kvp[1])}

\end{minted}

\cppitem{Remove a list elements from another list}
\begin{minted}
  [frame=single, rulecolor=blue, framesep=1mm, baselinestretch=1, fontsize=\footnotesize]{cpp}
watchlist=['SGO.PA', 'GLE.PA', 'LVMH.PA', 'MT.PA', 'ACA.PA']
banks=['GLE.PA', 'ACA.PA', 'BNP.PA']
watchlist=[ticker for ticker in watchlist if ticker not in banks]

\end{minted}

\cppitem{Check if a dictionary is included in another dictionary and get the difference}
\begin{minted}
  [frame=single, rulecolor=blue, framesep=1mm, baselinestretch=1, fontsize=\footnotesize]{cpp}
included=set(stocks_dict2.items()).issubset(set(stocks_dict1.items()))
diff={key: value for (key, value) in stocks_dict1.items() 
    if (key, value) not in stocks_dict2.items()}

\end{minted}

\cppitem{Find the matched object(s) in a list}
\begin{minted}
  [frame=single, rulecolor=blue, framesep=1mm, baselinestretch=1, fontsize=\footnotesize]{cpp}
# if the list contains a single match
match = next((s for s in stocks if s.ticker == equity.reuters_ticker), None)
# if the list contains one or more matches
matches = [s for s in stocks if s.ticker == equity.reuters_ticker]

\end{minted}

\cppitem{File basename}
\begin{minted}
  [frame=single, rulecolor=blue, framesep=1mm, baselinestretch=1, fontsize=\footnotesize]{cpp}
import os
s.path.basename(temp_tex_file.name)

\end{minted}

\cppitem{Capitalize the first letter of each word (for only the first letter of the string use capitalize())}
\begin{minted}
  [frame=single, rulecolor=blue, framesep=1mm, baselinestretch=1, fontsize=\footnotesize]{cpp}
s.title()

\end{minted}

\cppitem{Date in french}
\begin{minted}
  [frame=single, rulecolor=blue, framesep=1mm, baselinestretch=1, fontsize=\footnotesize]{cpp}
start_date.strftime("%B %Y").title() 
start_date.strftime('%d %B %Y').title()

\end{minted}

\cppitem{Iterate through 2 lists}
\begin{minted}
  [frame=single, rulecolor=blue, framesep=1mm, baselinestretch=1, fontsize=\footnotesize]{cpp}
for a,b in zip(l1, l2):
        print(a, b)

\end{minted}

\cppitem{Convert a list (of non string objects) to a string}
\begin{minted}
  [frame=single, rulecolor=blue, framesep=1mm, baselinestretch=1, fontsize=\footnotesize]{cpp}
', '.join(str(v) for v in start_date_range)
', '.join(map(str, list_of_ints))

\end{minted}

\cppitem{Range of date (frequency one month\, using pandas library)}
\begin{minted}
  [frame=single, rulecolor=blue, framesep=1mm, baselinestretch=1, fontsize=\footnotesize]{cpp}
pandas.period_range(start_date, end_date, freq='M')

\end{minted}

\cppitem{Add a delta to datetime}
\begin{minted}
  [frame=single, rulecolor=blue, framesep=1mm, baselinestretch=1, fontsize=\footnotesize]{cpp}
from datetime import datetime, timedelta
d = datetime.now() + timedelta(days=7)

\end{minted}

\cppitem{Run an external command}
\begin{minted}
  [frame=single, rulecolor=blue, framesep=1mm, baselinestretch=1, fontsize=\footnotesize]{cpp}
subprocess.run(["pdflatex", '-interaction=nonstopmode', temp_tex_file.name],
                       cwd='media/generated')

\end{minted}

\cppitem{Create a temporary file}
\begin{minted}
  [frame=single, rulecolor=blue, framesep=1mm, baselinestretch=1, fontsize=\footnotesize]{cpp}
temp_tex_file = tempfile.NamedTemporaryFile(mode="w", suffix=".tex", delete=False)

\end{minted}

\cppitem{Convert a number to word (using num2words library)}
\begin{minted}
  [frame=single, rulecolor=blue, framesep=1mm, baselinestretch=1, fontsize=\footnotesize]{cpp}
amount_words = num2words(amount, lang='fr')

\end{minted}

\cppitem{Access environment variable}
\begin{minted}
  [frame=single, rulecolor=blue, framesep=1mm, baselinestretch=1, fontsize=\footnotesize]{cpp}
mail_secret = os.environ['MAIL_SECRET']

\end{minted}

\cppitem{Log to a file and console}
\begin{minted}
  [frame=single, rulecolor=blue, framesep=1mm, baselinestretch=1, fontsize=\footnotesize]{cpp}
logging.basicConfig(
    level=logging.INFO,
    format="%(asctime)s [%(levelname)s] %(message)s",
    handlers=[
        logging.FileHandler("company_info.log"),
        logging.StreamHandler()
    ]
)

\end{minted}

\cppitem{Read file by (n) lines each iteration}
\begin{minted}
  [frame=single, rulecolor=blue, framesep=1mm, baselinestretch=1, fontsize=\footnotesize]{cpp}
  with open(filename, 'r') as infile:
    while True:
      next_n_lines = list(islice(infile, 5))
      if not next_n_lines:
          break

\end{minted}


\end{multicols}

\end{document}
