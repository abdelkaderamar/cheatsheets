\documentclass[10pt,a4paper]{article}
\usepackage[utf8]{inputenc}

\usepackage[landscape,margin=1cm]{geometry}
\usepackage[english]{babel}


% colour themes to come. KnitR?

%-------------------------

\title{\color{w3schools}Python Libraries Cheatsheet
}
%\author{Abdelkader Amar}
%\date{\today}

\usepackage{colourful_template}
% \input{colourful-template-commands.tex}


%--------------------------------------------------------------------------------
\begin{document}

\maketitle

\small
\begin{multicols}{3}

\thispagestyle{empty}
\scriptsize
% \tableofcontents



\begin{codebox}{python}{argparse}
    # Argument expecting one argument
    parser.add_argument('--indice', '-i', default='CAC 40', action='store', dest='indices', help='Process indices')
    # python argparse_sample.py -i CAC\ 40
    # Namespace(auto_create=False, equities=None, indices='CAC 40')
    # Arguments can be repeated
    parser.add_argument('--equity', '-e', action='append', dest='equities', help='Process equities')
    # python argparse_sample.py -i CAC\ 40 -e 'GLE.PA' -e 'SGOB.PA'
    # Namespace(auto_create=False, equities=['GLE.PA', 'SGOB.PA'], indices='CAC 40')
    # Argument without a value (boolean value)
    parser.add_argument('--auto-create', default=False, action='store_true', dest='auto_create', help='Automatically create equity')
    # python argparse_sample.py -i CAC\ 40 -e 'GLE.PA' -e 'SGOB.PA' --auto-create
    # Namespace(auto_create=True, equities=['GLE.PA', 'SGOB.PA'], indices='CAC 40')
    # Argument without a typed value
    parser.add_argument('--interval', default=30, action='store', dest='interval', help='Interval time')
    # python argparse_sample.py
    # Namespace(auto_create=False, equities=None, indices='CAC 40', interval=30)
    # python argparse_sample.py --interval=22
    # Namespace(auto_create=False, equities=None, indices='CAC 40', interval='22')
    # python argparse_sample.py --interval 32
    # Namespace(auto_create=False, equities=None, indices='CAC 40', interval='32')

\end{codebox}

\begin{codebox}{python}{pyyaml}
    import yaml
    # load multi-documents file
    with open('multi-doc.yaml', 'r') as f:
    docs = yaml.load_all(f, Loader=yaml.FullLoader)
    for doc in docs:
      pass
    # load document file
    with open('doc.yaml', 'r') as f:
    doc = yaml.load(f, Loader=yaml.FullLoader)

\end{codebox}


%---------------------------------------------
\AtNextBibliography{\footnotesize}
\printbibliography  
\end{multicols}

\end{document}
