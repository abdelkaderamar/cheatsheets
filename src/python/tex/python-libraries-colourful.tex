\documentclass[10pt,a4paper]{article}
\usepackage[utf8]{inputenc}

\usepackage[landscape,margin=1cm]{geometry}
\usepackage[english]{babel}


% colour themes to come. KnitR?

%-------------------------

\title{\color{w3schools}Python Libraries Cheatsheet
}
%\author{Abdelkader Amar}
%\date{\today}

\usepackage{colourful_template}
% \usepackage[default]{raleway}
\usepackage{fontawesome}
\usepackage[T1]{fontenc}

\usepackage{hyperref}
\usepackage{enumitem}
\usepackage{lipsum}

\usepackage{xcolor}
\definecolor{customcolor}{HTML}{616AC5}
\definecolor{alert}{HTML}{CD5C5C}
\definecolor{w3schools}{HTML}{4CAF50}
\definecolor{subbox}{gray}{0.60}
\definecolor{codecolor}{HTML}{FFC300}
\colorlet{xx}{customcolor}


%--------------------------Editor mode.

\usepackage
[citestyle=authoryear,
sorting=nty,	  		%Sorts bibliography by year, name, title
autocite=footnote, 		%Autocite command generates footnotes
autolang=hyphen, 		
mincrossrefs=1, 	
backend=biber]
{biblatex}

\DeclareFieldFormat{postnote}{#1}
\DeclareFieldFormat{multipostnote}{#1}
\DeclareAutoCiteCommand{footnote}[f]{\footcite}{\footcites}

\bibliography{literature}
%----------------------------------------
%--------------------------------------------------------------------------------
\usepackage{tcolorbox}

\tcbuselibrary{most,listingsutf8,minted}

\tcbset{tcbox width=auto,left=1mm,top=1mm,bottom=1mm,
right=1mm,boxsep=1mm,middle=1pt}

\newenvironment{mycolorbox}[2]{%
\begin{tcolorbox}[grow to left by=-1em,grow to right by=-1em,capture=minipage,fonttitle=\large\bfseries, enhanced jigsaw,boxsep=1mm,colback=#1!30!white,on line,tcbox width=auto, toptitle=0mm,colframe=#1,opacityback=0.7,nobeforeafter,title=#2]%
}{\end{tcolorbox}\\[0.2em]}

\newenvironment{subbox}[2]{%
\begin{tcolorbox}[capture=minipage,fonttitle=\normalsize\bfseries, enhanced jigsaw,boxsep=1mm,colback=#1!30!white,on line,tcbox width=auto,left=0.3em,top=1mm, toptitle=0mm,colframe=#1,opacityback=0.7,nobeforeafter,title=#2]\footnotesize %
}{\normalsize\end{tcolorbox}\vspace{0.1em}}

\newenvironment{multibox}[1]{%
\begin{tcbraster}[raster columns=#1,raster equal height,nobeforeafter,raster column skip=1em,raster left skip=1em,raster right skip=1em]}{\end{tcbraster}}

\newenvironment{textbox}[1]{\begin{mycolorbox}{customcolor}{#1}}{\end{mycolorbox}}

%-------------------------------
\newtcblisting{codebox}[2]{colback=codecolor!5,colframe=codecolor!80!black,listing only, 
minted options={numbers=left,style=tcblatex,fontsize=\tiny,breaklines,autogobble,linenos,numbersep=3mm},
left=5mm,enhanced,
title=#2, fonttitle=\bfseries,
listing engine=minted,minted language=#1}

%--------------------------------------------------------------------------------
\newcommand{\punkti}{~\lbrack\dots\rbrack~}

\renewenvironment{quote}
               {\list{\faQuoteLeft\phantom{ }}{\rightmargin\leftmargin}%
                \item\relax\scriptsize\ignorespaces}
               {\unskip\unskip\phantom{xx}\faQuoteRight\endlist}
               

%--------------------------------------------------------------------------------
\newcommand{\bgupper}[3]{\colorbox{#1}{\color{#2}\huge\bfseries\MakeUppercase{#3}}}
\newcommand{\bg}[3]{\colorbox{#1}{\bfseries\color{#2}#3}}

\newcommand{\mycommand}[2]{{\ttfamily\detokenize{#1}}~\dotfill{}~{\footnotesize #2}\\}
\newcommand{\sep}{{\scriptsize~\faCircle{ }~}}


\newcommand{\bggreen}[1]{\medskip\bgupper{w3schools}{black}{#1}\\[0.5em]}
\newcommand{\green}[1]{\smallskip\bg{w3schools}{white}{#1}\\}
\newcommand{\red}[1]{\smallskip\bg{alert}{white}{#1}\\}

\usepackage{multicol}
\setlength{\columnsep}{30pt}

\setlength{\parindent}{0pt}
\pagestyle{empty}

\usepackage{csquotes}

\newcommand{\loremipsum}{Lorem ipsum dolor sit amet.}


%--------------------------------------------------------------------------------
\begin{document}

\maketitle

\small
\begin{multicols}{3}

\thispagestyle{empty}
\scriptsize
% \tableofcontents



\begin{codebox}{python}{argparse}
    # Argument expecting one argument
    parser.add_argument('--indice', '-i', default='CAC 40', action='store', dest='indices', help='Process indices')
    # python argparse_sample.py -i CAC\ 40
    # Namespace(auto_create=False, equities=None, indices='CAC 40')
    # Arguments can be repeated
    parser.add_argument('--equity', '-e', action='append', dest='equities', help='Process equities')
    # python argparse_sample.py -i CAC\ 40 -e 'GLE.PA' -e 'SGOB.PA'
    # Namespace(auto_create=False, equities=['GLE.PA', 'SGOB.PA'], indices='CAC 40')
    # Argument without a value (boolean value)
    parser.add_argument('--auto-create', default=False, action='store_true', dest='auto_create', help='Automatically create equity')
    # python argparse_sample.py -i CAC\ 40 -e 'GLE.PA' -e 'SGOB.PA' --auto-create
    # Namespace(auto_create=True, equities=['GLE.PA', 'SGOB.PA'], indices='CAC 40')
    # Argument without a typed value
    parser.add_argument('--interval', default=30, action='store', dest='interval', help='Interval time')
    # python argparse_sample.py
    # Namespace(auto_create=False, equities=None, indices='CAC 40', interval=30)
    # python argparse_sample.py --interval=22
    # Namespace(auto_create=False, equities=None, indices='CAC 40', interval='22')
    # python argparse_sample.py --interval 32
    # Namespace(auto_create=False, equities=None, indices='CAC 40', interval='32')

\end{codebox}

\begin{codebox}{python}{pyyaml}
    import yaml
    # load multi-documents file
    with open('multi-doc.yaml', 'r') as f:
    docs = yaml.load_all(f, Loader=yaml.FullLoader)
    for doc in docs:
      pass
    # load document file
    with open('doc.yaml', 'r') as f:
    doc = yaml.load(f, Loader=yaml.FullLoader)

\end{codebox}


%---------------------------------------------
\AtNextBibliography{\footnotesize}
\printbibliography  
\end{multicols}

\end{document}
