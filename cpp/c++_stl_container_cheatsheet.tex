\documentclass[10pt,english,landscape, a4]{article}
\usepackage{cpp}


\pdfinfo{
  /Title (C++ STL Container Cheatsheet.pdf)
  /Creator (Abdelkader Amar)
  /Author (Abdelkader Amar)
  /Subject (C++)
  /Keywords (C++, STL, Programming, Latex)}



\begin{document}
\raggedright\

\begin{center}
  \cheatsheettitle{C++ STL Containers Cheatsheet}
\end{center}
%\vspace{0.5cm}

\footnotesize
\begin{multicols}{3}
  \raggedcolumns
  \noindent    %<---- here

% \section{Function return type deduction}
%     \texttt{auto f()}
% \begin{minted}
%     [frame=single,
%     framesep=2mm,
%     baselinestretch=1.2,
%     fontsize=\footnotesize,
%     ]{cpp}
% auto f_cpp14() { return 1; }
%
% auto f_cpp11() -> int { return 1; }
% \end{minted}

%-------------------------------------------------------------------------------

\cppitem{Sort elements in a container}
\begin{minted}
  [frame=single, rulecolor=blue, framesep=1mm, baselinestretch=1, fontsize=\footnotesize]{cpp}
// vector
vector<int> v1{5, 3, 10, 2};
sort(v1.begin(), v1.end());

// array
array<int, 5> a1{3, 2, 10, 5, 2};
sort(a1.begin(), a1.end());

// deque
deque<int> dq1{3, 5, 1, 10, 7};
sort(dq1.begin(), dq1.end());

// forward_list
forward_list<int> fl1{33, 5, 1, 10, 7};
// error : no random iterator
f// sort(fl1.begin(), fl2.end()); 
l1.sort();

// List
list<int> l1{5, 3, 10, 2};
// error std::sort needs random iterator
// sort(l1.begin(), l1.end()); 
l1.sort();
\end{minted}

%-------------------------------------------------------------------------------

%-------------------------------------------------------------------------------
\columnbreak
%-------------------------------------------------------------------------------

\cppitem{Defining a sort comparator}
\begin{minted}
  [frame=single, rulecolor=blue, framesep=1mm, baselinestretch=1, fontsize=\footnotesize]{cpp}
struct point_t {
    int x, y;
};

vector<point_t> v1 { {2, 3}, {5, 6}, {1, 2}};

// error => no comparaison operator
sort(v1.begin(), v1.end()); 

// method 1: operator < 
bool operator<(const point_t& p1, const point_t& p2) {
    if (p1.x == p2.x) return p1.y < p2.y;
    return p1.x < p2.x;
}

// method 2: operator <
struct point_t {
    int x, y;

    bool operator < (const point_t& other) {
        if (x == other.x) return y < other.y;
        return x < other.x;
    }
};

// method 3: sort predicate
struct point_comparator
{
  bool operator()(const point_t& p1, 
                  const point_t& p2)
  {
      if (p1.x == p2.x) return p1.y < p2.y;
      return p1.x < p2.x;
  }
};
sort(v1.begin(), v1.end(), point_comparator());

// method 4: using lambda expressions
sort(v1.begin(), v1.end(),
    [](const point_t &p1, const point_t &p2)
     {
       if (p1.x == p2.x)
          return p1.y < p2.y;
       return p1.x < p2.x;
   });

\end{minted}

%-------------------------------------------------------------------------------

\cppitem{.....}
\begin{minted}
  [frame=single, rulecolor=blue, framesep=1mm, baselinestretch=1, fontsize=\footnotesize]{cpp}


\end{minted}

%-------------------------------------------------------------------------------


%-------------------------------------------------------------------------------

\columnbreak

%-------------------------------------------------------------------------------

\cppitem{Print the content of a container}
\begin{minted}
  [frame=single, rulecolor=blue, framesep=1mm, baselinestretch=1, fontsize=\footnotesize]{cpp}
    vector<point_t> v1 { {2, 3}, {5, 6}, {1, 2}};
// C++11    
copy(v1.begin(), v1.end(), 
     ostream_iterator<point_t>(cout, " "));
cout << endl;
// C++20
ranges::copy(v1, 
             ostream_iterator<point_t>(cout, " "));
cout << endl;

\end{minted}

%-------------------------------------------------------------------------------

\cppitem{-------}
\begin{minted}
  [frame=single, rulecolor=blue, framesep=1mm, baselinestretch=1, fontsize=\footnotesize]{cpp}

\end{minted}

%-------------------------------------------------------------------------------
\cppitem{-------}
\begin{minted}
  [frame=single, rulecolor=blue, framesep=1mm, baselinestretch=1, fontsize=\footnotesize]{cpp}

\end{minted}

%-------------------------------------------------------------------------------
\cppitem{-------}
\begin{minted}
  [frame=single, rulecolor=blue, framesep=1mm, baselinestretch=1, fontsize=\footnotesize]{cpp}

\end{minted}

\cppitem{-------}
\begin{minted}
  [frame=single, rulecolor=blue, framesep=1mm, baselinestretch=1, fontsize=\footnotesize]{cpp}

\end{minted}

%-------------------------------------------------------------------------------

\cppitem{-------}
\begin{minted}
  [frame=single, rulecolor=blue, framesep=1mm, baselinestretch=1, fontsize=\footnotesize]{cpp}

\end{minted}

%-------------------------------------------------------------------------------
\cppitem{-------}
\begin{minted}
  [frame=single, rulecolor=blue, framesep=1mm, baselinestretch=1, fontsize=\footnotesize]{cpp}

\end{minted}

%-------------------------------------------------------------------------------
\cppitem{-------}
\begin{minted}
  [frame=single, rulecolor=blue, framesep=1mm, baselinestretch=1, fontsize=\footnotesize]{cpp}

\end{minted}

%-------------------------------------------------------------------------------

% \cppitem{Feature}
% \begin{minted}
%   [frame=single, rulecolor=blue, framesep=1mm, baselinestretch=1, fontsize=\footnotesize]{cpp}
% \end{minted}
%

\end{multicols}

\end{document}
