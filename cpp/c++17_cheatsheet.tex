\documentclass[10pt,english,landscape, a4]{article}
\usepackage{amarcs}


\pdfinfo{
  /Title (C++17 Cheatsheet.pdf)
  /Creator (Latex)
  /Author (Abdelkader Amar)
  /Subject (C++)
  /Keywords (C++, C++17, Programming, Latex)}



\begin{document}
\raggedright\

\begin{center}
  \cheatsheettitle{C++17 Language New Features Cheatsheet}
\end{center}
%\vspace{0.5cm}

\footnotesize
\begin{multicols}{3}
  \raggedcolumns
  \noindent    %<---- here

% \section{Function return type deduction}
%     \texttt{auto f()}
% \begin{minted}
%     [frame=single,
%     framesep=2mm,
%     baselinestretch=1.2,
%     fontsize=\footnotesize,
%     ]{cpp}
% auto f_cpp14() { return 1; }
%
% auto f_cpp11() -> int { return 1; }
% \end{minted}

%-------------------------------------------------------------------------------

\cppitem{Template argument deduction for class templates}
\begin{minted}
  [frame=single, rulecolor=blue, framesep=1mm, baselinestretch=1, fontsize=\footnotesize]{cpp}
pair p1(1, 2.0);
// vs
pair<int, double> p2(1, 2.0);
\end{minted}

%-------------------------------------------------------------------------------

\cppitem{Declaring non-type template parameters with auto}
\begin{minted}
  [frame=single, rulecolor=blue, framesep=1mm, baselinestretch=1, fontsize=\footnotesize]{cpp}
template <auto ... seq>
struct my_integer_sequence {
  // Implementation here ...
};

// Explicitly pass type `int` as template argument.
auto seq = std::integer_sequence<int, 0, 1, 2>();
// Type is deduced to be `int`.
auto seq2 = my_integer_sequence<0, 1, 2>();
\end{minted}

%-------------------------------------------------------------------------------

\cppitem{Folding expressions}
\begin{minted}
  [frame=single, rulecolor=blue, framesep=1mm, baselinestretch=1, fontsize=\footnotesize]{cpp}
template<typename ... Ts>
auto sum_fold_exp(const Ts& ... ts) {
  return (ts + ...);
}

template<typename ... Ts>
auto print_fold(const Ts& ... ts)
{
  ((cout << ts << " "), ... );
}
\end{minted}

%-------------------------------------------------------------------------------

\cppitem{New rules for auto deduction from braced-init-list}
\begin{minted}
  [frame=single, rulecolor=blue, framesep=1mm, baselinestretch=1, fontsize=\footnotesize]{cpp}
// error: not a single element
auto x1{ 1, 2, 3 };

// decltype(x2) is std::initializer_list<int>
auto x2 = { 1, 2, 3 };

// decltype(x3) is int, previously deduced to
// initializer_list<int>
auto x3{ 3 };

// decltype(x4) is double
auto x4{ 3.0 };
\end{minted}

%-------------------------------------------------------------------------------

\columnbreak

%-------------------------------------------------------------------------------

\cppitem{constexpr lambda}
\begin{minted}
  [frame=single, rulecolor=blue, framesep=1mm, baselinestretch=1, fontsize=\footnotesize]{cpp}
auto identity = [] (int n) constexpr { return n; };
static_assert(identity(123) == 123);

constexpr int addOne(int n) {
  return [n] { return n + 1; }();
}
static_assert(addOne(1) == 2);
\end{minted}

%-------------------------------------------------------------------------------

\cppitem{Lambda capture \texttt{this} by value}
\begin{minted}
  [frame=single, rulecolor=blue, framesep=1mm, baselinestretch=1, fontsize=\footnotesize]{cpp}
struct foo
{
  foo() : _x{0} {}
  int _x;
  auto log_by_ref() {
    return [this]() { cout << _x << endl; };
  }
  auto log_by_val() {
    return [*this]() { cout << _x << endl; };
  }

};

int main(int argc, char *agrv[])
{
  struct foo f;
  auto ref = f.log_by_ref();
  auto val = f.log_by_val();
  f._x = 1234;
  ref();
  val();
  f._x = 4321;
  ref();
  val();
}
\end{minted}

%-------------------------------------------------------------------------------

\cppitem{Inline variables}
\begin{minted}
  [frame=single, rulecolor=blue, framesep=1mm, baselinestretch=1, fontsize=\footnotesize]{cpp}
struct S { int x; };
inline S x1 = S{321};
\end{minted}

%-------------------------------------------------------------------------------


%-------------------------------------------------------------------------------

\cppitem{Nested namespaces}
\begin{minted}
  [frame=single, rulecolor=blue, framesep=1mm, baselinestretch=1, fontsize=\footnotesize]{cpp}
namespace A::B::C {
  class foo;
}
\end{minted}

\columnbreak

%-------------------------------------------------------------------------------

\cppitem{Structured bindings}
\begin{minted}
  [frame=single, rulecolor=blue, framesep=1mm, baselinestretch=1, fontsize=\footnotesize]{cpp}
template<typename T>
pair<T, bool> racine(T d) {
  if (d<0) return pair(-1, false);
  return pair(sqrt(d), true);
}

auto [s, success] = racine(1998.0);
if (success) cout << s << endl;
\end{minted}

%-------------------------------------------------------------------------------

\cppitem{Selection statements with initializer}
\begin{minted}
  [frame=single, rulecolor=blue, framesep=1mm, baselinestretch=1, fontsize=\footnotesize]{cpp}
if (auto res=m.insert({key,value}); res.second) {
  cout<<key<<"/"<<value<<" inserted"<<endl;
}
\end{minted}

%-------------------------------------------------------------------------------

\cppitem{constexpr if}
\begin{minted}
  [frame=single, rulecolor=blue, framesep=1mm, baselinestretch=1, fontsize=\footnotesize]{cpp}
template <typename T> int compute(T x) {
  // no () around consexpr
  if constexpr (std::is_integral<T>::value) {
    return x * x;
  } else if constexpr (is_same<T, string>::value) {
    return x.size();
  } else if constexpr (is_base_of<foo, T>::value) {
    x.bar();
    return 0;
  }
  return 0;
}
\end{minted}

%-------------------------------------------------------------------------------

\cppitem{UTF-8 Character Literals}
\begin{minted}
  [frame=single, rulecolor=blue, framesep=1mm, baselinestretch=1, fontsize=\footnotesize]{cpp}
char x = u8'x';
\end{minted}

%-------------------------------------------------------------------------------

\cppitem{Direct List Initialization of Enums}
\begin{minted}
  [frame=single, rulecolor=blue, framesep=1mm, baselinestretch=1, fontsize=\footnotesize]{cpp}
// underlying type must be fixed (char here)
enum class color : char { red, blue, green };
// must be non-narrowing, i.e 129 is an error
color c1 { 3 }, c2 { 88 };
\end{minted}

%-------------------------------------------------------------------------------

% \cppitem{Feature}
% \begin{minted}
%   [frame=single, rulecolor=blue, framesep=1mm, baselinestretch=1, fontsize=\footnotesize]{cpp}
% \end{minted}
%

\end{multicols}

\end{document}
