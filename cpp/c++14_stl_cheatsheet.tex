\documentclass[10pt,english,landscape, a4]{article}
\usepackage{multicol}
\usepackage{calc}
\usepackage[landscape]{geometry}
\usepackage{color,graphicx,overpic}
%\usepackage{xcolor}

\usepackage[T1]{fontenc}
\usepackage[bitstream-charter]{mathdesign}
\usepackage[utf8]{inputenc}
\usepackage{url}
\usepackage{amsfonts}
\usepackage{array,booktabs}
\usepackage{textcomp}
\usepackage[usenames,dvipsnames,table]{xcolor}
\usepackage[most]{tcolorbox}
\usepackage{tabularx}
\usepackage{multirow}
\usepackage{colortbl}
\usepackage{tikz}
\usepackage{environ}
\usepackage{listings}

\usepackage{minted}
\usemintedstyle{borland}

\pdfinfo{
  /Title (C++14 Cheatsheet.pdf)
  /Creator (Latex)
  /Author (Abdelkader Amar)
  /Subject (C++)
  /Keywords (C++, C++14, Programming, Latex)}

\usetikzlibrary{calc}
\pgfdeclarelayer{background}
\pgfdeclarelayer{foreground}
\pgfsetlayers{background,main,foreground}

\geometry{top=-0.5cm,left=1cm,right=1cm,bottom=1cm}

\pagestyle{empty} % Turn off header and footer

% \renewcommand\rmdefault{phv} % Arial
% \renewcommand\sfdefault{phv} % Arial

% Redefine section commands to use less space
\makeatletter
\renewcommand{\section}{\@startsection{section}{1}{0mm}%
{-1ex plus -.5ex minus -.2ex}%
{0.5ex plus .2ex}%x
{\normalfont\large\bfseries}}
\renewcommand{\subsection}{\@startsection{subsection}{2}{0mm}%
{-1explus -.5ex minus -.2ex}%
{0.5ex plus .2ex}%
{\normalfont\normalsize\bfseries}}
\renewcommand{\subsubsection}{\@startsection{subsubsection}{3}{0mm}%
{-1ex plus -.5ex minus -.2ex}%
{1ex plus .2ex}%
{\normalfont\small\bfseries}}
\makeatother

\setcounter{secnumdepth}{0} % Don't print section numbers
\setlength{\parindent}{0pt}
\setlength{\parskip}{0pt plus 0.5ex}

\definecolor{TableHead}{rgb}{0.353, 0.329, 0.667}
\definecolor{TableRow}{rgb}{0.816, 0.812, 0.902}

\NewEnviron{keys}[1]{
% \begin{center}
\smallskip
\begin{tikzpicture}
  \rowcolors{1}{}{TableRow}
  \centering
  \node (tbl) [inner sep=0pt] {
  \begin{tabular}{p{4.2cm}p{3.25cm}}
    \rowcolor{TableHead}
    \multicolumn{2}{l}{\normalsize\textbf{\color{white}{#1}}}\parbox{0pt}{\rule{0pt}{0.3ex+\baselineskip}}\\
    \BODY
    \arrayrulecolor{TableHead}\specialrule{.17em}{0em}{.2em}
    \end{tabular}};
    \begin{pgfonlayer}{background}
      \draw[rounded corners=2pt,top color=TableHead,bottom color=TableHead, draw=white]
      ($(tbl.north west)-(0,-0.05)$) rectangle ($(tbl.north east)-(0.0,0.15)$);
      \draw[rounded corners=2pt,top color=TableHead,bottom color=TableHead, draw=white]
      ($(tbl.south west)-(0.0,-0.11)$) rectangle ($(tbl.south east)-(-0.0,-0.02)$);
    \end{pgfonlayer}
  \end{tikzpicture}
  % \end{center}
  }

\newcommand{\feature}[2]{
\begin{center}
\smallskip
\begin{tikzpicture}
  \rowcolors{1}{}{TableRow}
  \centering
  \node (tbl) [inner sep=0pt] {
  \begin{tabular}{p{13cm}}
    \rowcolor{TableHead}
    \multicolumn{1}{l}{\normalsize\textbf{\color{white}{#1}}}\parbox{0pt}{\rule{0pt}{0.3ex+\baselineskip}}\\
    #2 \\
    \arrayrulecolor{TableHead}\specialrule{.17em}{0em}{.1em}
    \end{tabular}};
    \begin{pgfonlayer}{background}
      \draw[rounded corners=2pt,top color=TableHead,bottom color=TableHead, draw=white]
      ($(tbl.north west)-(0,-0.05)$) rectangle ($(tbl.north east)-(0.0,0.15)$);
      \draw[rounded corners=2pt,top color=TableHead,bottom color=TableHead, draw=white]
      ($(tbl.south west)-(0.0,-0.11)$) rectangle ($(tbl.south east)-(-0.0,-0.02)$);
    \end{pgfonlayer}
  \end{tikzpicture}
  \end{center}
  }

\newcommand{\cppitem}[1]{
  \vspace{0.15cm}
    \begin{tikzpicture}
      \node [fill=TableHead, text=white, draw, rounded corners=2pt, minimum height=0.55cm]{\normalsize\textbf{#1}};
    \end{tikzpicture}
    \vspace{0.02cm}
  }

\definecolor{myblue}{cmyk}{1,.72,0,.38}

\makeatletter
\renewcommand{\section}{\@startsection{section}{1}{0mm}%
                                  {.2ex}%
                                  {.2ex}%x
                                  {\color{myblue}\sffamily\small\bfseries}}
\renewcommand{\subsection}{\@startsection{subsection}{1}{0mm}%
                                  {.2ex}%
                                  {.2ex}%x
                                  {\sffamily\bfseries}}
\begin{document}
\raggedright\

\begin{center}
  \Huge{\underline{C++14 STL Cheatsheet}}
\end{center}
\vspace{1cm}

\footnotesize
\begin{multicols}{3}
  \raggedcolumns
  \noindent    %<---- here

% \section{Function return type deduction}
%     \texttt{auto f()}
% \begin{minted}
%     [frame=single,
%     framesep=2mm,
%     baselinestretch=1.2,
%     fontsize=\footnotesize,
%     ]{cpp}
% auto f_cpp14() { return 1; }
%
% auto f_cpp11() -> int { return 1; }
% \end{minted}

%-------------------------------------------------------------------------------

\cppitem{\texttt{shared\_timed\_mutex}}
\begin{minted}
  [frame=single, rulecolor=blue, framesep=1mm, baselinestretch=1, fontsize=\footnotesize]{cpp}
\end{minted}

%-------------------------------------------------------------------------------

\cppitem{\texttt{shared\_lock}}
\begin{minted}
  [frame=single, rulecolor=blue, framesep=1mm, baselinestretch=1, fontsize=\footnotesize]{cpp}
\end{minted}

%-------------------------------------------------------------------------------

\cppitem{Standard user-defined literals}
\begin{minted}
  [frame=single, rulecolor=blue, framesep=1mm, baselinestretch=1, fontsize=\footnotesize]{cpp}
\end{minted}

%-------------------------------------------------------------------------------

\cppitem{Tuple addressing via type}
\begin{minted}
  [frame=single, rulecolor=blue, framesep=1mm, baselinestretch=1, fontsize=\footnotesize]{cpp}
\end{minted}

%-------------------------------------------------------------------------------

\cppitem{\texttt{is\_final}}
\begin{minted}
  [frame=single, rulecolor=blue, framesep=1mm, baselinestretch=1, fontsize=\footnotesize]{cpp}
\end{minted}

%-------------------------------------------------------------------------------

\cppitem{\texttt{equal}}
\begin{minted}
  [frame=single, rulecolor=blue, framesep=1mm, baselinestretch=1, fontsize=\footnotesize]{cpp}
\end{minted}

%-------------------------------------------------------------------------------
\cppitem{\texttt{mismatch}}
\begin{minted}
  [frame=single, rulecolor=blue, framesep=1mm, baselinestretch=1, fontsize=\footnotesize]{cpp}
\end{minted}

%-------------------------------------------------------------------------------
\cppitem{\texttt{is\_permutation}}
\begin{minted}
  [frame=single, rulecolor=blue, framesep=1mm, baselinestretch=1, fontsize=\footnotesize]{cpp}
\end{minted}

%-------------------------------------------------------------------------------
\cppitem{\texttt{make\_unique}}
\begin{minted}
  [frame=single, rulecolor=blue, framesep=1mm, baselinestretch=1, fontsize=\footnotesize]{cpp}
\end{minted}

%-------------------------------------------------------------------------------
\cppitem{\texttt{integral\_constants}}
\begin{minted}
  [frame=single, rulecolor=blue, framesep=1mm, baselinestretch=1, fontsize=\footnotesize]{cpp}
\end{minted}

%-------------------------------------------------------------------------------
\cppitem{\texttt{integral\_sequence}}
\begin{minted}
  [frame=single, rulecolor=blue, framesep=1mm, baselinestretch=1, fontsize=\footnotesize]{cpp}
\end{minted}

%-------------------------------------------------------------------------------
\cppitem{\texttt{cbegin}, \texttt{cend}, \texttt{crbegin} and \texttt{crend}}
\begin{minted}
  [frame=single, rulecolor=blue, framesep=1mm, baselinestretch=1, fontsize=\footnotesize]{cpp}
\end{minted}

%-------------------------------------------------------------------------------
\cppitem{\texttt{exchange}}
\begin{minted}
  [frame=single, rulecolor=blue, framesep=1mm, baselinestretch=1, fontsize=\footnotesize]{cpp}
\end{minted}

%-------------------------------------------------------------------------------
\cppitem{Heteregeneous lookup in associative container}
\begin{minted}
  [frame=single, rulecolor=blue, framesep=1mm, baselinestretch=1, fontsize=\footnotesize]{cpp}
\end{minted}

%-------------------------------------------------------------------------------
\cppitem{}
\begin{minted}
  [frame=single, rulecolor=blue, framesep=1mm, baselinestretch=1, fontsize=\footnotesize]{cpp}
\end{minted}

%-------------------------------------------------------------------------------
\cppitem{}
\begin{minted}
  [frame=single, rulecolor=blue, framesep=1mm, baselinestretch=1, fontsize=\footnotesize]{cpp}
\end{minted}

%-------------------------------------------------------------------------------

\columnbreak

% \cppitem{Feature}
% \begin{minted}
%   [frame=single, rulecolor=blue, framesep=1mm, baselinestretch=1, fontsize=\footnotesize]{cpp}
% \end{minted}
%


\end{multicols}

\end{document}
