\documentclass[10pt,english,landscape, a4]{article}
\usepackage{cpp}


\pdfinfo{
  /Title (C++ STL Container Cheatsheet.pdf)
  /Creator (Abdelkader Amar)
  /Author (Abdelkader Amar)
  /Subject (C++)
  /Keywords (C++, STL, Programming, Latex)}



\begin{document}
\raggedright\

\begin{center}
  \cheatsheettitle{C++ STL Containers Cheatsheet
}
\end{center}
%\vspace{0.5cm}

\footnotesize
\begin{multicols}{3}
  \raggedcolumns
  \noindent    %<---- here


\cppitem{Sort elements in a container}
\begin{minted}
  [frame=single, rulecolor=blue, framesep=1mm, baselinestretch=1, fontsize=\footnotesize]{cpp}
// vector
vector<int> v1{5, 3, 10, 2};
sort(v1.begin(), v1.end());
// array
array<int, 5> a1{3, 2, 10, 5, 2};
sort(a1.begin(), a1.end());
// deque
deque<int> dq1{3, 5, 1, 10, 7};
sort(dq1.begin(), dq1.end());
// forward_list
forward_list<int> fl1{33, 5, 1, 10, 7};
// error : no random iterator
f// sort(fl1.begin(), fl2.end());
l1.sort();
// List
list<int> l1{5, 3, 10, 2};
// error std::sort needs random iterator
// sort(l1.begin(), l1.end());
l1.sort();

\end{minted}

\cppitem{Defining a sort comparator}
\begin{minted}
  [frame=single, rulecolor=blue, framesep=1mm, baselinestretch=1, fontsize=\footnotesize]{cpp}
struct point_t {
    int x, y;
};
vector<point_t> v1 { {2, 3}, {5, 6}, {1, 2}};
// error => no comparaison operator
sort(v1.begin(), v1.end());
// method 1: operator <
bool operator<(const point_t& p1, const point_t& p2) {
    if (p1.x == p2.x) return p1.y < p2.y;
    return p1.x < p2.x;
}
// method 2: operator <
struct point_t {
    int x, y;
    bool operator < (const point_t& other) {
        if (x == other.x) return y < other.y;
        return x < other.x;
    }
};
// method 3: sort predicate
struct point_comparator
{
  bool operator()(const point_t& p1,
                  const point_t& p2)
  {
      if (p1.x == p2.x) return p1.y < p2.y;
      return p1.x < p2.x;
  }
};
sort(v1.begin(), v1.end(), point_comparator());
// method 4: using lambda expressions
sort(v1.begin(), v1.end(),
    [](const point_t &p1, const point_t &p2)
     {
       if (p1.x == p2.x)
          return p1.y < p2.y;
       return p1.x < p2.x;
   });

\end{minted}

\cppitem{Sort container in descending order}
\begin{minted}
  [frame=single, rulecolor=blue, framesep=1mm, baselinestretch=1, fontsize=\footnotesize]{cpp}
std::sort(container.begin(), container.end(),
          std::greater<>());

\end{minted}

\cppitem{Print the content of a container}
\begin{minted}
  [frame=single, rulecolor=blue, framesep=1mm, baselinestretch=1, fontsize=\footnotesize]{cpp}
// C++11
copy(v1.begin(), v1.end(),
     ostream_iterator<point_t>(cout, " "));
cout << endl;
// C++20
ranges::copy(v1,
             ostream_iterator<point_t>(cout, " "));
cout << endl;

\end{minted}

\cppitem{Concatenating two vectors/list/deque}
\begin{minted}
  [frame=single, rulecolor=blue, framesep=1mm, baselinestretch=1, fontsize=\footnotesize]{cpp}
c1.insert( c1.end(), c2.begin(), c2.end() );

\end{minted}

\cppitem{Concatenating two sets/map}
\begin{minted}
  [frame=single, rulecolor=blue, framesep=1mm, baselinestretch=1, fontsize=\footnotesize]{cpp}
// valid also for unordered_set/unordered_map
c1.insert(c2.begin(), c2.end());

\end{minted}

\cppitem{Concatenate two forward\_list}
\begin{minted}
  [frame=single, rulecolor=blue, framesep=1mm, baselinestretch=1, fontsize=\footnotesize]{cpp}
fl1.merge(fl2);

\end{minted}

\cppitem{Loop through map}
\begin{minted}
  [frame=single, rulecolor=blue, framesep=1mm, baselinestretch=1, fontsize=\footnotesize]{cpp}
// With C++11
for (auto const& p : m)
{
  cout << p.first << "," << p.second << endl;
}
// With C++17
for (auto const& [k, v] : m) {
  cout << k << "," << v << endl;
}

\end{minted}

\cppitem{Sum of container elements}
\begin{minted}
  [frame=single, rulecolor=blue, framesep=1mm, baselinestretch=1, fontsize=\footnotesize]{cpp}
// for vector/list/forward_list/deque/set (be
// careful about initial value type)
std::accumulate(c1.begin(), c1.end(), 0);
// to sum custom types
struct point_t {
    int x, y;
    point_t operator+(const point_t& t) {
        return point_t{x+t.x, y+t.y};
    }
};
vector<point_t> v1{{1,2}, {2,3}, {3,4}};
point_t sum = accumulate(v1.begin(), v1.end(),
                        point_t{});

\end{minted}


\end{multicols}

\end{document}
