\documentclass[10pt,english,landscape, a4]{article}
\usepackage{multicol}
\usepackage{calc}
\usepackage[landscape]{geometry}
\usepackage{color,graphicx,overpic}

\usepackage[T1]{fontenc}
\usepackage[bitstream-charter]{mathdesign}
\usepackage[utf8]{inputenc}
\usepackage{url}
\usepackage{amsfonts}
\usepackage{array,booktabs}
\usepackage{textcomp}
\usepackage[usenames,dvipsnames,table]{xcolor}
\usepackage[most]{tcolorbox}
\usepackage{tabularx}
\usepackage{multirow}
\usepackage{colortbl}
\usepackage{tikz}
\usepackage{environ}
\usepackage{listings}

\usepackage{minted}
\usemintedstyle{borland}

\pdfinfo{
  /Title (C++14 Cheatsheet.pdf)
  /Creator (Latex)
  /Author (Abdelkader Amar)
  /Subject (C++)
  /Keywords (C++, C++14, Programming, Latex)}

\usetikzlibrary{calc}
\pgfdeclarelayer{background}
\pgfdeclarelayer{foreground}
\pgfsetlayers{background,main,foreground}

\geometry{top=-0.5cm,left=1cm,right=1cm,bottom=1cm}

\pagestyle{empty} % Turn off header and footer

% \renewcommand\rmdefault{phv} % Arial
% \renewcommand\sfdefault{phv} % Arial

% Redefine section commands to use less space
\makeatletter
\renewcommand{\section}{\@startsection{section}{1}{0mm}%
{-1ex plus -.5ex minus -.2ex}%
{0.5ex plus .2ex}%x
{\normalfont\large\bfseries}}
\renewcommand{\subsection}{\@startsection{subsection}{2}{0mm}%
{-1explus -.5ex minus -.2ex}%
{0.5ex plus .2ex}%
{\normalfont\normalsize\bfseries}}
\renewcommand{\subsubsection}{\@startsection{subsubsection}{3}{0mm}%
{-1ex plus -.5ex minus -.2ex}%
{1ex plus .2ex}%
{\normalfont\small\bfseries}}
\makeatother

\setcounter{secnumdepth}{0} % Don't print section numbers
\setlength{\parindent}{0pt}
\setlength{\parskip}{0pt plus 0.5ex}

\definecolor{TableHead}{rgb}{0.353, 0.329, 0.667}
\definecolor{TableRow}{rgb}{0.816, 0.812, 0.902}

\NewEnviron{keys}[1]{
% \begin{center}
\smallskip
\begin{tikzpicture}
  \rowcolors{1}{}{TableRow}
  \centering
  \node (tbl) [inner sep=0pt] {
  \begin{tabular}{p{4.2cm}p{3.25cm}}
    \rowcolor{TableHead}
    \multicolumn{2}{l}{\normalsize\textbf{\color{white}{#1}}}\parbox{0pt}{\rule{0pt}{0.3ex+\baselineskip}}\\
    \BODY
    \arrayrulecolor{TableHead}\specialrule{.17em}{0em}{.2em}
    \end{tabular}};
    \begin{pgfonlayer}{background}
      \draw[rounded corners=2pt,top color=TableHead,bottom color=TableHead, draw=white]
      ($(tbl.north west)-(0,-0.05)$) rectangle ($(tbl.north east)-(0.0,0.15)$);
      \draw[rounded corners=2pt,top color=TableHead,bottom color=TableHead, draw=white]
      ($(tbl.south west)-(0.0,-0.11)$) rectangle ($(tbl.south east)-(-0.0,-0.02)$);
    \end{pgfonlayer}
  \end{tikzpicture}
  % \end{center}
  }

\newcommand{\feature}[2]{
\begin{center}
\smallskip
\begin{tikzpicture}
  \rowcolors{1}{}{TableRow}
  \centering
  \node (tbl) [inner sep=0pt] {
  \begin{tabular}{p{13cm}}
    \rowcolor{TableHead}
    \multicolumn{1}{l}{\normalsize\textbf{\color{white}{#1}}}\parbox{0pt}{\rule{0pt}{0.3ex+\baselineskip}}\\
    #2 \\
    \arrayrulecolor{TableHead}\specialrule{.17em}{0em}{.1em}
    \end{tabular}};
    \begin{pgfonlayer}{background}
      \draw[rounded corners=2pt,top color=TableHead,bottom color=TableHead, draw=white]
      ($(tbl.north west)-(0,-0.05)$) rectangle ($(tbl.north east)-(0.0,0.15)$);
      \draw[rounded corners=2pt,top color=TableHead,bottom color=TableHead, draw=white]
      ($(tbl.south west)-(0.0,-0.11)$) rectangle ($(tbl.south east)-(-0.0,-0.02)$);
    \end{pgfonlayer}
  \end{tikzpicture}
  \end{center}
  }

\newcommand{\cppitem}[1]{
  \vspace{0.15cm}
    \begin{tikzpicture}
      \node [fill=TableHead, text=white, draw, rounded corners=2pt, minimum height=0.55cm]{\normalsize\textbf{#1}};
    \end{tikzpicture}
    \vspace{0.02cm}
  }

\definecolor{myblue}{cmyk}{1,.72,0,.38}

\makeatletter
\renewcommand{\section}{\@startsection{section}{1}{0mm}%
                                  {.2ex}%
                                  {.2ex}%x
                                  {\color{myblue}\sffamily\small\bfseries}}
\renewcommand{\subsection}{\@startsection{subsection}{1}{0mm}%
                                  {.2ex}%
                                  {.2ex}%x
                                  {\sffamily\bfseries}}
\begin{document}
\raggedright\

\begin{center}
  \Huge{\underline{C++14 Cheatsheet}}
\end{center}
\vspace{1cm}

\footnotesize
\begin{multicols}{2}
  \raggedcolumns
  \noindent    %<---- here

% \section{Function return type deduction}
%     \texttt{auto f()}
% \begin{minted}
%     [frame=single,
%     framesep=2mm,
%     baselinestretch=1.2,
%     fontsize=\footnotesize,
%     ]{cpp}
% auto f_cpp14() { return 1; }
%
% auto f_cpp11() -> int { return 1; }
% \end{minted}

%-------------------------------------------------------------------------------

\cppitem{Function return type deduction}
\begin{minted}
  [frame=single, rulecolor=blue, framesep=1mm, baselinestretch=1, fontsize=\footnotesize]{cpp}
auto f_cpp14() { return 1; } // return type deduced to int

auto f_cpp11() -> int { return 1; } // trailing return type is needed in c++11
\end{minted}

%-------------------------------------------------------------------------------

\cppitem{Variable templates}
\begin{minted}[frame=single, rulecolor=blue, framesep=1mm, baselinestretch=1, fontsize=\footnotesize]{cpp}
template<typename T>
constexpr T pi = T(3.141592653589793238462643383);

template<>
constexpr const char* pi<const char*> = "pi";

cout << pi<long double> << endl;
cout << pi<double> << endl;
cout << pi<float> << endl;
cout << pi<const char*> << endl;
\end{minted}

%-------------------------------------------------------------------------------

\cppitem{Aggreagte member initialization}
\begin{minted}
  [frame=single, rulecolor=blue, framesep=1mm, baselinestretch=1, fontsize=\footnotesize]{cpp}
struct foo_2 {
  int x{10}, y, z;
};
struct foo_2 f2 { 1, 2 }; // error in c++11

struct foo_3 {
  int x{1}, y{2}, z{3};
};
struct foo_3 f3_a { 1, 2 }; // error in c++11
\end{minted}

%-------------------------------------------------------------------------------

\cppitem{Binary literals}
\begin{minted}
  [frame=single, rulecolor=blue, framesep=1mm, baselinestretch=1, fontsize=\footnotesize]{cpp}
short i = 0b0101010101010101; // 0b
short j = 0B1101010101010101; // or 0B
\end{minted}

%-------------------------------------------------------------------------------

\cppitem{The attribute deprecated}
\begin{minted}
  [frame=single, rulecolor=blue, framesep=1mm, baselinestretch=1, fontsize=\footnotesize]{cpp}
class foo
{
public:
  // with message
  [[deprecated("x is not protected. Use getter instead")]]
  int x;
};
// without message
[[deprecated]] void f() {};
[[deprecated("g will not be supported from next release")]] void g() {};
\end{minted}

%-------------------------------------------------------------------------------

\columnbreak

%-------------------------------------------------------------------------------

\cppitem{Digit separator}
\begin{minted}
  [frame=single, rulecolor=blue, framesep=1mm, baselinestretch=1, fontsize=\footnotesize]{cpp}
int i = 1'928'229'292; // single quote placed arbitrarily
int j = 1928'2'292'92; // for integer and floating point literals
\end{minted}

%-------------------------------------------------------------------------------

\cppitem{Generic lambda}
\begin{minted}
  [frame=single, rulecolor=blue, framesep=1mm, baselinestretch=1, fontsize=\footnotesize]{cpp}
auto lambda = [](auto x, auto y) { return x + y; };

cout << lambda(1, 2) << endl; // x and y deduced to int
cout << lambda(1.6, 2.5) << endl; // x and y deduced to float
string s1{"1"}, s2{".5"};
cout << lambda(s1, s2) << endl; // x and y deduced to std::string
\end{minted}

%-------------------------------------------------------------------------------

\cppitem{Relaxed constexpr restrictions}

constexpr functions may contain:
\begin{itemize}
  \item declarations (except *static* or *thread\_local*)
  \item *if* and *switch*
  \item loops
\end{itemize}
\begin{minted}
  [frame=single, rulecolor=blue, framesep=1mm, baselinestretch=1, fontsize=\footnotesize]{cpp}
constexpr int f_cpp14(int x) {
  if (x % 2 == 0)
    return x * 10;
  return x;
}
\end{minted}

%-------------------------------------------------------------------------------

\cppitem{Lambda capture expression}
\begin{minted}
  [frame=single, rulecolor=blue, framesep=1mm, baselinestretch=1, fontsize=\footnotesize]{cpp}
auto lambda_constant = [value = 3](int x) { return value * x; };

auto lambda_func_call = [value = f(argc)](int x) { return value * x; };

string s{"foo"};
auto lambda_move = [value = move(s)](int x) { return value + to_string(x); };

\end{minted}

%-------------------------------------------------------------------------------

\cppitem{Alternate type deduction on declaration}
\begin{minted}
  [frame=single, rulecolor=blue, framesep=1mm, baselinestretch=1, fontsize=\footnotesize]{cpp}
int i;
int&& f();
auto x3a = i;                  // decltype(x3a) is int
decltype(auto) x3d = i;        // decltype(x3d) is int
auto x4a = (i);                // decltype(x4a) is int
decltype(auto) x4d = (i);      // decltype(x4d) is int&
auto x5a = f();                // decltype(x5a) is int
decltype(auto) x5d = f();      // decltype(x5d) is int&&
auto x6a = { 1, 2 };           // decltype(x6a) is std::initializer_list<int>
decltype(auto) x6d = { 1, 2 }; // error, { 1, 2 } is not an expression
auto *x7a = &i;                // decltype(x7a) is int*
decltype(auto)*x7d = &i;   // error, declared type is not plain decltype(auto)
\end{minted}


%-------------------------------------------------------------------------------

% \cppitem{Feature}
% \begin{minted}
%   [frame=single, rulecolor=blue, framesep=1mm, baselinestretch=1, fontsize=\footnotesize]{cpp}
% \end{minted}
%

\end{multicols}

\end{document}
