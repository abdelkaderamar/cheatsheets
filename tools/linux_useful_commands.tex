\documentclass[10pt,english,landscape, a4]{article}
\usepackage{linux}


\pdfinfo{
  /Title (Latex Cheatsheet.pdf)
  /Creator (Abdelkader Amar)
  /Author (Abdelkader Amar)
  /Subject (Linux)
  /Keywords (Linux)}



\begin{document}
\raggedright\

\begin{center}
  \cheatsheettitle{Linux Useful Commands}
\end{center}
%\vspace{0.5cm}

\footnotesize
\begin{multicols}{3}
  \raggedcolumns
  \noindent    %<---- here

\begin{center}{\section{Text Files}}\end{center}
  
%-------------------------------------------------------------------------------

\linuxitem{Interleave Lines of Two Text Files}
\begin{minted}
  [frame=single, rulecolor=blue, framesep=1mm, baselinestretch=1, fontsize=\footnotesize]{bash}
paste -d '\n' file1 file2
\end{minted}

%-------------------------------------------------------------------------------
\columnbreak
%-------------------------------------------------------------------------------

\linuxitem{exiftool - Renaming images}
Renames all images in directory "DIR" according to the individual file's creation date in the form "YYYYmmdd_HHMMSS.ext".
\begin{minted}
  [frame=single, rulecolor=blue, framesep=1mm, baselinestretch=1, fontsize=\footnotesize]{bash}
exiftool "-FileName<CreateDate" -d "%Y%m%d_%H%M%S.%%e" DIR
\end{minted}


%-------------------------------------------------------------------------------
\columnbreak
%-------------------------------------------------------------------------------

\linuxitem{exiftool - Moving images}
Moves all images originally in directory "DIR" into a directory hierarchy organized by year/month/day
\begin{minted}
  [frame=single, rulecolor=blue, framesep=1mm, baselinestretch=1, fontsize=\footnotesize]{bash}
exiftool "-Directory<DateTimeOriginal" -d "%Y/%m/%d" DIR
\end{minted}


%-------------------------------------------------------------------------------

% \linuxitem{Feature}
% \begin{minted}
%   [frame=single, rulecolor=blue, framesep=1mm, baselinestretch=1, fontsize=\footnotesize]{tex}
% \end{minted}
%

\end{multicols}

\end{document}

%bibtex : Garder la casse du titre d'une entrée : entourer de {}
