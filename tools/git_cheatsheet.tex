\documentclass[10pt,english,landscape]{article}
\usepackage{multicol}
\usepackage{calc}
\usepackage[landscape]{geometry}
\usepackage{color,graphicx,overpic}

\usepackage[T1]{fontenc}
\usepackage[bitstream-charter]{mathdesign}
\usepackage[utf8]{inputenc}
\usepackage{url}
\usepackage{amsfonts}
\usepackage{array,booktabs}
\usepackage{textcomp}
\usepackage[usenames,dvipsnames,table]{xcolor}
\usepackage[most]{tcolorbox}
\usepackage{tabularx}
\usepackage{multirow}
\usepackage{colortbl}
\usepackage{tikz}
\usepackage{environ}

\usetikzlibrary{calc}
\pgfdeclarelayer{background}
\pgfdeclarelayer{foreground}
\pgfsetlayers{background,main,foreground}

\geometry{top=-0.5cm,left=1cm,right=1cm,bottom=1cm}

\pagestyle{empty} % Turn off header and footer

% \renewcommand\rmdefault{phv} % Arial
% \renewcommand\sfdefault{phv} % Arial

% Redefine section commands to use less space
\makeatletter
\renewcommand{\section}{\@startsection{section}{1}{0mm}%
{-1ex plus -.5ex minus -.2ex}%
{0.5ex plus .2ex}%x
{\normalfont\large\bfseries}}
\renewcommand{\subsection}{\@startsection{subsection}{2}{0mm}%
{-1explus -.5ex minus -.2ex}%
{0.5ex plus .2ex}%
{\normalfont\normalsize\bfseries}}
\renewcommand{\subsubsection}{\@startsection{subsubsection}{3}{0mm}%
{-1ex plus -.5ex minus -.2ex}%
{1ex plus .2ex}%
{\normalfont\small\bfseries}}
\makeatother

\setcounter{secnumdepth}{0} % Don't print section numbers
\setlength{\parindent}{0pt}
\setlength{\parskip}{0pt plus 0.5ex}

\definecolor{TableHead}{rgb}{0.353, 0.329, 0.667}
\definecolor{TableRow}{rgb}{0.816, 0.812, 0.902}

\NewEnviron{keys}[1]{
% \begin{center}
\smallskip
\begin{tikzpicture}
  \rowcolors{1}{}{TableRow}
  \centering
  \node (tbl) [inner sep=0pt] {
  \begin{tabular}{p{4cm}p{7cm}}
    \rowcolor{TableHead}
    \multicolumn{2}{l}{\normalsize\textbf{\color{white}{#1}}}\parbox{0pt}{\rule{0pt}{0.3ex+\baselineskip}}\\
    \BODY
    \arrayrulecolor{TableHead}\specialrule{.17em}{0em}{.2em}
    \end{tabular}};
    \begin{pgfonlayer}{background}
      \draw[rounded corners=2pt,top color=TableHead,bottom color=TableHead, draw=white]
      ($(tbl.north west)-(0,-0.05)$) rectangle ($(tbl.north east)-(0.0,0.15)$);
      \draw[rounded corners=2pt,top color=TableHead,bottom color=TableHead, draw=white]
      ($(tbl.south west)-(0.0,-0.11)$) rectangle ($(tbl.south east)-(-0.0,-0.02)$);
    \end{pgfonlayer}
  \end{tikzpicture}
  % \end{center}
  }

  \begin{document}

  \raggedright\

  \begin{center}
    \Huge{\underline{GIT Cheatsheet}}
  \end{center}

  \footnotesize
  \begin{multicols}{2}
    \raggedcolumns
    \noindent    %<---- here

    \centering\section{Working With Repositories}

    \begin{keys}{Basic commands}
      Local repository status                & \texttt{git status} \\
      Stage a file/dir to commit list        & \texttt{git add <file or dir>} \\
      Commit (with text editor)              & \texttt{git commit} \\
      Commit (with commit message)           & \texttt{git commit -m "message"} \\
      Send committed modification to remote repository (master branch) & \texttt{git push origin master} \\
      Verbose mode                           &
      \texttt{git command --verbose} or \newline \texttt{git command -v} \\
      Dry run                                &
      \texttt{git command --dry-run} \\
      Unstage a file                         &
      \texttt{git rm --cached <file> \\
    \end{keys}

    \begin{keys}{Repository status}
      Unpublished commits           & \texttt{git log origin/master..HEAD} \\
      Display diff of unpublished commits  & \texttt{git diff origin/master} \\
      Display diff of unpublished commits  & \texttt{git diff origin/master..HEAD} \\
      Remote repository information     & \texttt{git remote show origin} \\
    \end{keys}

   \centering\section{Github}

   \begin{keys}{Access}
     Change repo URL from https to ssh (replace user by the real username) &
     \texttt{git remote set-url origin git+ssh://git@github.com/user/reponame.git} \\
     Text                 & \texttt{Command} \\
     Text                 & \texttt{Command} \\
     Text                 & \texttt{Command} \\
     Text                 & \texttt{Command} \\
     Text                 & \texttt{Command} \\
   \end{keys}


    \columnbreak

    \centering\section{Working With Files}

    \begin{keys}{Commit}
      Modify a commit message           &
      \texttt{git commit --amend} \\
      Undo commit &
      \texttt{git commit -m "message" \newline
      git reset HEAD\~                              \newline
      << edit files as necessary >>                \newline
      git add <...>                                \newline
      git commit -c ORIG\_HEAD
      } \\
      Revert one or more commits (by hashes) &
      \texttt{git revert a867b4af 25eee4ca 0766c053} \\
      Revert the last two commits &
      \texttt{git revert HEAD~2..HEAD} \\
      Revert range of commits (by hashes interval) &
      \texttt{git revert a867b4af..0766c053} \\
    \end{keys}
%
    \centering\section{Useful Tips}

    \begin{keys}{Mirroring}
      Push to multiple Git repositories &
      \texttt{git remote add github \textbackslash \newline
      \hspace*{0.5cm} git@github.com:aamar/my-project.git \newline
      git remote add bb \textbackslash \newline
      \hspace*{0.5cm} git@bitbucket.org:aamar/my-project.git \newline
      git remote set-url ---add ---push origin \textbackslash \newline
      \hspace*{0.5cm} git@github.com:aamar/my-project.git \newline
      git remote set-url ---add ---push origin \textbackslash \newline
      \hspace*{0.5cm} git@bitbucket.org:aamar/my-project.git
      } \\
    \end{keys}
%
%    \begin{keys}{Something}
%      Text                 & \texttt{Command} \\
%      Text                 & \texttt{Command} \\
%      Text                 & \texttt{Command} \\
%      Text                 & \texttt{Command} \\
%      Text                 & \texttt{Command} \\
%      Text                 & \texttt{Command} \\
%    \end{keys}
%
%    \columnbreak
%
%    \centering\section{Section}
%
%    \begin{keys}{Something}
%      Text                 & \texttt{Command} \\
%      Text                 & \texttt{Command} \\
%      Text                 & \texttt{Command} \\
%      Text                 & \texttt{Command} \\
%      Text                 & \texttt{Command} \\
%      Text                 & \texttt{Command} \\
%    \end{keys}
%
%    \begin{keys}{Something}
%      Text                 & \texttt{Command} \\
%      Text                 & \texttt{Command} \\
%      Text                 & \texttt{Command} \\
%      Text                 & \texttt{Command} \\
%      Text                 & \texttt{Command} \\
%      Text                 & \texttt{Command} \\
%    \end{keys}
%

  \end{multicols}

  \end{document}
