\documentclass[10pt,english,landscape]{article}
\usepackage{multicol}
\usepackage{calc}
\usepackage[landscape]{geometry}
\usepackage{color,graphicx,overpic}

\usepackage[T1]{fontenc}
\usepackage[bitstream-charter]{mathdesign}
\usepackage[utf8]{inputenc}
\usepackage{url}
\usepackage{amsfonts}
\usepackage{array,booktabs}
\usepackage{textcomp}
\usepackage[usenames,dvipsnames,table]{xcolor}
\usepackage[most]{tcolorbox}
\usepackage{tabularx}
\usepackage{multirow}
\usepackage{colortbl}
\usepackage{tikz}
\usepackage{environ}

\usetikzlibrary{calc}
\pgfdeclarelayer{background}
\pgfdeclarelayer{foreground}
\pgfsetlayers{background,main,foreground}

\geometry{top=-0.5cm,left=1cm,right=1cm,bottom=1cm}

\pagestyle{empty} % Turn off header and footer

% \renewcommand\rmdefault{phv} % Arial
% \renewcommand\sfdefault{phv} % Arial

% Redefine section commands to use less space
\makeatletter
\renewcommand{\section}{\@startsection{section}{1}{0mm}%
{-1ex plus -.5ex minus -.2ex}%
{0.5ex plus .2ex}%x
{\normalfont\large\bfseries}}
\renewcommand{\subsection}{\@startsection{subsection}{2}{0mm}%
{-1explus -.5ex minus -.2ex}%
{0.5ex plus .2ex}%
{\normalfont\normalsize\bfseries}}
\renewcommand{\subsubsection}{\@startsection{subsubsection}{3}{0mm}%
{-1ex plus -.5ex minus -.2ex}%
{1ex plus .2ex}%
{\normalfont\small\bfseries}}
\makeatother

\setcounter{secnumdepth}{0} % Don't print section numbers
\setlength{\parindent}{0pt}
\setlength{\parskip}{0pt plus 0.5ex}

\definecolor{TableHead}{rgb}{0.353, 0.329, 0.667}
\definecolor{TableRow}{rgb}{0.816, 0.812, 0.902}

\NewEnviron{keys}[1]{
% \begin{center}
\smallskip
\begin{tikzpicture}
  \rowcolors{1}{}{TableRow}
  \centering
  \node (tbl) [inner sep=0pt] {
  \begin{tabular}{p{2.45cm}p{5cm}}
    \rowcolor{TableHead}
    \multicolumn{2}{l}{\normalsize\textbf{\color{white}{#1}}}\parbox{0pt}{\rule{0pt}{0.3ex+\baselineskip}}\\
    \BODY
    \arrayrulecolor{TableHead}\specialrule{.17em}{0em}{.2em}
    \end{tabular}};
    \begin{pgfonlayer}{background}
      \draw[rounded corners=2pt,top color=TableHead,bottom color=TableHead, draw=white]
      ($(tbl.north west)-(0,-0.05)$) rectangle ($(tbl.north east)-(0.0,0.15)$);
      \draw[rounded corners=2pt,top color=TableHead,bottom color=TableHead, draw=white]
      ($(tbl.south west)-(0.0,-0.11)$) rectangle ($(tbl.south east)-(-0.0,-0.02)$);
    \end{pgfonlayer}
  \end{tikzpicture}
  % \end{center}
  }

  \begin{document}

  \raggedright\

  \begin{center}
    \Huge{\underline{Materialize Cheatsheet}}
  \end{center}

  \footnotesize
  \begin{multicols}{3}
    \raggedcolumns
    \noindent    %<---- here

    \centering\section{CSS}

    \begin{keys}{Container}
      Container            & \texttt{<div class="container">} \\
      Colonne              & \texttt{<div class="col s1">} \\
      Ligne                & \texttt{<div class="row">} \\
      Offset               & \texttt{<div class="col s6 offset-s6">} \\
      Colonne Mobile       & \texttt{<div class="col s12">} \\
      Colonne Tablette     & \texttt{<div class="col m12">} \\
      Colonne Desktop      & \texttt{<div class="col l12">} \\
    \end{keys}

    \begin{keys}{Alignement}
      Alignement vertical  & \texttt{<div class="valign-wrapper">} \\
      Texte à gauche       & \texttt{<h5 class="left-align">This should be left aligned</h5>} \\
      Texte à droite       & \texttt{<h5 class="right-align">This should be right aligned</h5>} \\
      Texte centré         & \texttt{<h5 class="center-align">This should be center aligned</h5>} \\
      Texte tronqué        & \texttt{<h4 class="truncate">This is an extremely long title that will be truncated</h4>} \\
    \end{keys}

    \centering\section{Section}

    \begin{keys}{Something}
      Diviseur             & \texttt{<div class="divider"></div>} \\
      Section              & \texttt{<div class="section">} \\
      Text                 & \texttt{Command} \\
      Text                 & \texttt{Command} \\
      Text                 & \texttt{Command} \\
      Text                 & \texttt{Command} \\
    \end{keys}


    \columnbreak

    \centering\section{Section}

    \begin{keys}{Something}
      Text                 & \texttt{Command} \\
      Text                 & \texttt{Command} \\
      Text                 & \texttt{Command} \\
      Text                 & \texttt{Command} \\
      Text                 & \texttt{Command} \\
      Text                 & \texttt{Command} \\
    \end{keys}

    \begin{keys}{Something}
      Text                 & \texttt{Command} \\
      Text                 & \texttt{Command} \\
      Text                 & \texttt{Command} \\
      Text                 & \texttt{Command} \\
      Text                 & \texttt{Command} \\
      Text                 & \texttt{Command} \\
    \end{keys}

    \begin{keys}{Something}
      Text                 & \texttt{Command} \\
      Text                 & \texttt{Command} \\
      Text                 & \texttt{Command} \\
      Text                 & \texttt{Command} \\
      Text                 & \texttt{Command} \\
      Text                 & \texttt{Command} \\
    \end{keys}

    \columnbreak
    \centering\section{Section}

    \begin{keys}{Something}
      Text                 & \texttt{Command} \\
      Text                 & \texttt{Command} \\
      Text                 & \texttt{Command} \\
      Text                 & \texttt{Command} \\
      Text                 & \texttt{Command} \\
      Text                 & \texttt{Command} \\
    \end{keys}

    \begin{keys}{Something}
      Text                 & \texttt{Command} \\
      Text                 & \texttt{Command} \\
      Text                 & \texttt{Command} \\
      Text                 & \texttt{Command} \\
      Text                 & \texttt{Command} \\
      Text                 & \texttt{Command} \\
    \end{keys}


  \end{multicols}

  \end{document}
