\documentclass[10pt,english,landscape, a4]{article}
\usepackage{texcheatsheet}


\pdfinfo{
  /Title (Latex Cheatsheet.pdf)
  /Creator (Abdelkader Amar)
  /Author (Abdelkader Amar)
  /Subject (Latex)
  /Keywords (Latex)}



\begin{document}
\raggedright\

\begin{center}
  \cheatsheettitle{Latex Cheatsheet}
\end{center}
%\vspace{0.5cm}

\footnotesize
\begin{multicols}{3}
  \raggedcolumns
  \noindent    %<---- here

%-------------------------------------------------------------------------------

\centering\section{Text properties}

\begin{keys}{Font}
  \textrm{Roman} &
  \textbackslash textrm\{Roman\} \\
  \textsf{Sans serif} &
  \textbackslash textsf\{Sans serif\} \\
  \texttt{Typewriter} &
  \textbackslash texttt\{Typewriter\} \\
  \textmd{Medium series} &
  \textbackslash textmd\{Medium series\} \\
  \textbf{Bold series} &
  \textbackslash textbf\{Bold series\} \\
  \textup{Upright shape} &
  \textbackslash textup\{Upright shape\} \\
  \textit{Italic shape} &
  \textbackslash textit\{Italic shape\} \\
  \textsl{Slanted shape} &
  \textbackslash textsl\{Slanted shape\} \\
  \textsc{Small Caps shape} &
  \textbackslash textsc\{Small Caps shape\} \\
  \emph{Emphasized} &
  \textbackslash emph\{Emphasized\} \\
  \textnormal{Document font} &
  \textbackslash textnormal\{Document font\} \\
  \underline{Underline} &
  \textbackslash underline\{Underline\} \\
\end{keys}

\texitem{Text Size}
\begin{minted}
  [frame=single, rulecolor=blue, framesep=1mm, baselinestretch=1, fontsize=\footnotesize]{tex}
\tiny \scriptsize \footnotesize \small \normalsize
\large \Large \LARGE \huge  \Huge
\end{minted}

%-------------------------------------------------------------------------------
\columnbreak
%-------------------------------------------------------------------------------

\texitem{Feature}
\begin{minted}
  [frame=single, rulecolor=blue, framesep=1mm, baselinestretch=1, fontsize=\footnotesize]{tex}
\end{minted}


%-------------------------------------------------------------------------------
\columnbreak
%-------------------------------------------------------------------------------

\texitem{Feature}
\begin{minted}
  [frame=single, rulecolor=blue, framesep=1mm, baselinestretch=1, fontsize=\footnotesize]{tex}
\end{minted}


%-------------------------------------------------------------------------------

\texitem{Multiple columns}
\begin{minted}
  [frame=single, rulecolor=blue, framesep=1mm, baselinestretch=1, fontsize=\footnotesize]{tex}
\usepackage{multicol}

% column separation set to 1cm
\setlength{\columnsep}{1cm}

% insert a blue vertical rulers (add color package)
\setlength{\columnseprule}{1pt}
\def\columnseprulecolor{\color{blue}}

% use multicols* for unbalanced columns
\begin{multicols}{3}
["Header text", which is inserted in between square
brackets. This is optional and will be displayed on
top of the multicolumn text.]

% insert a column breakpoint
\columnbreak
  
\end{multicols}
\end{minted}


\texitem{Multiple line in some cells: p-type column}
\begin{minted}
  [frame=single, rulecolor=blue, framesep=1mm, baselinestretch=1, fontsize=\footnotesize]{tex}

\begin{tabular}{l|p{15mm}}
  \hline
  foo & bar \newline rlz \\
  \hline
\end{tabular}

\end{minted}

\texitem{Multiple line in some cells: tabular within tabular}
\begin{minted}
  [frame=single, rulecolor=blue, framesep=1mm, baselinestretch=1, fontsize=\footnotesize]{tex}
\begin{tabular}{cccc}
  One & Two & Three & Four \\
  Een & Twee & Drie & Vier \\
  One & Two & 
  \begin{tabular}{@{}c@{}}
    Three \\ Drie\end{tabular}
  & Four
\end{tabular}
\end{minted}

\texitem{Multiple line in some cells: \textbackslash{}shortstack}
\begin{minted}
  [frame=single, rulecolor=blue, framesep=1mm, baselinestretch=1, fontsize=\footnotesize]{tex}
\begin{tabular}{ccc}
    one & two & three \\
    one & two & \shortstack{aa \\ bb}\\

\end{tabular}
\end{minted}

%-------------------------------------------------------------------------------

% \texitem{Feature}
% \begin{minted}
%   [frame=single, rulecolor=blue, framesep=1mm, baselinestretch=1, fontsize=\footnotesize]{tex}
% \end{minted}
%

\end{multicols}

\end{document}

%bibtex : Garder la casse du titre d'une entrée : entourer de {}
